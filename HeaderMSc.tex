\documentclass[12pt,spanish,fleqn,openany,letterpaper,pagesize]{scrbook} %report, scrbook

\usepackage[ansinew]{inputenc}
\usepackage[english]{babel}
\usepackage{dingbat}
\usepackage{fancyhdr}
\usepackage{epsfig}
\usepackage{epic}
\usepackage{eepic}
\usepackage{amsmath}
\usepackage{threeparttable}
\usepackage{amscd}
\usepackage{here}
\usepackage{graphicx}
\usepackage{lscape}
\usepackage{tabularx}
\usepackage{subfigure}
\usepackage{longtable}
\usepackage{tabularx,colortbl}
\usepackage{color}
\usepackage{xcolor}

\usepackage{titlesec}
\setcounter{tocdepth}{3}
\setcounter{secnumdepth}{3}

%tables
\usepackage{pbox}
\usepackage{longtable} % para tablas largas
\usepackage{lipsum}
\usepackage{ragged2e}
% centering cells
\newcolumntype{P}[1]{>{\centering\arraybackslash}p{#1}}
%Numbers with circle:
\usepackage{tikz}
\newcommand*\circled[1]{\tikz[baseline=(char.base)]{
            \node[shape=circle,draw,inner sep=2pt] (char) {#1};}}

%algorithm2e and comment style
\usepackage[ruled,vlined,linesnumbered]{algorithm2e}
\newenvironment{english_algorithm}[1][htb]
  {\renewcommand{\algorithmcfname}{Algorithm}% Update algorithm name
   \begin{algorithm}[#1]%
  }{\end{algorithm}}

%///////////////////////////////////
%\usepackage[utf8]{inputenc}
% Hyper-references
\usepackage[unicode=true, bookmarks=true,bookmarksnumbered=true,bookmarksopen=false, breaklinks=true, pdfborder={0 0 0},backref=page,colorlinks=true]{hyperref}
\hypersetup{pdftitle={Master Dissertation Intelligent Probing for SDN Monitoring}, pdfauthor={Edwin Ferney Castillo Quintero}, colorlinks=true, citecolor=blue,linkcolor=black, urlcolor=green}

\usepackage{rotating} %Para rotar texto, objetos y tablas seite. No se ve en DVI solo en PS. Seite 328 Hundebuch
                        %se usa junto con \rotate, \sidewidestable ....

\usepackage{svg}

\renewcommand{\theequation}{\thechapter-\arabic{equation}}
\renewcommand{\thefigure}{\textbf{\thechapter-\arabic{figure}}}
\renewcommand{\thetable}{\textbf{\thechapter-\arabic{table}}}


\pagestyle{fancyplain}%\addtolength{\headwidth}{\marginparwidth}

%color linea del header
%\renewcommand{\headrulewidth}{2pt}% 2pt header rule
%\renewcommand{\headrule}{\hbox to\headwidth{%
%  \color{red}\leaders\hrule height \headrulewidth\hfill}}

%\oddsidemarginDenota el margen izquierdo de una p ́agina impar
%.\evensidemarginDenota el margen izquierdo en una p ́agina par
%http://ima.udg.es/Docencia/3105200736/tema9p.pdf
%\textheight22.5cm \topmargin0cm \textwidth16.5cm
\textheight22.0cm \topmargin0cm \textwidth16.5cm
\oddsidemargin0.5cm \evensidemargin-0.5cm%

\renewcommand{\chaptermark}[1]{\markboth{\thechapter\; #1}{}}
\renewcommand{\sectionmark}[1]{\markright{\thesection\; #1}}

\lhead[\fancyplain{}{\thepage}]{\fancyplain{}{\rightmark}}
\rhead[\fancyplain{}{\leftmark}]{\fancyplain{}{\thepage}}
\fancyfoot{}
\thispagestyle{fancy}%



\addtolength{\headwidth}{0cm}
\unitlength1mm %Define la unidad LE para Figuras
\mathindent0cm %Define la distancia de las formulas al texto,  fleqn las descentra
\marginparwidth0cm
\parindent0cm %Define la distancia de la primera linea de un parrafo a la margen

%Para tablas,  redefine el backschlash en tablas donde se define la posici\'{o}n del texto en las
%casillas (con \centering \raggedright o \raggedleft)
\newcommand{\PreserveBackslash}[1]{\let\temp=\\#1\let\\=\temp}
\let\PBS=\PreserveBackslash

%Espacio entre lineas
\renewcommand{\baselinestretch}{1.1}

%Neuer Befehl f\"{u}r die Tabelle Eigenschaften der Aktivkohlen
\newcommand{\arr}[1]{\raisebox{1.5ex}[0cm][0cm]{#1}}

%Neue Kommandos
\usepackage{Befehle}


%Trennungsliste
\hyphenation {Q-learning swit-ches}

\usepackage{pbsi}
\usepackage[T1]{fontenc}

\usepackage{datetime}
\newdateformat{monthyeardate}{%
  \monthname[\THEMONTH] \THEYEAR}

\newdateformat{yeardateonly}{%
  \yearname \THEYEAR} 
  
%Command             10pt    11pt    12pt
%\tiny               5       6       6
%\scriptsize         7       8       8
%\footnotesize       8       9       10
%\small              9       10      10.95
%\normalsize         10      10.95   12
%\large              12      12      14.4
%\Large              14.4    14.4    17.28
%\LARGE              17.28   17.28   20.74
%\huge               20.74   20.74   24.88
%\Huge               24.88   24.88   24.88

%enumerate list with circle
\newcommand*\circled[1]{\kern-2.5em%
  \put(0,4){\color{blue}\circle*{18}}\put(0,4){\circle{16}}%
  \put(-3,0){\color{white}\bfseries\large#1}~~}
\usepackage{enumitem}

%Including pages from PDF documents - https://texblog.org/2011/10/26/including-pages-from-pdf-documents/
\usepackage{pdfpages}
