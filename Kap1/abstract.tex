\textbf{\LARGE Abstract}\\\\
%\addcontentsline{toc}{chapter}{Abstract}\\\\
Traffic Monitoring assists in achieving the stability of networks by observing and quantifying their behavior. A proper traffic monitoring solution requires the accurate and timely collection of flow statistics. Many approaches have been proposed to monitor Software-Defined Networks. However, these approaches have diverse shortcomings. First, they are unconcerned about the trade-off between the probing interval and the Monitoring Accuracy (MA). Second, they lack intelligent mechanisms intended to optimize this trade-off by learning from network behavior. This master dissertation introduces an approach, called IPro, to address these shortcomings. IPro is formed by an architecture that follows the Knowledge-Defined Networking paradigm, an algorithm based on Reinforcement Learning, and an IPro prototype. In particular, IPro uses Reinforcement Learning to determine the probing interval that keeps within thresholds (target values) the Control Channel Overhead (CCO) and the Extra CPU Usage of the Controller (CUC). An extensive quantitative evaluation corroborates that IPro is an efficient approach for SDN Monitoring regarding CCO, CCU, and MA.\\[2.0cm]


\textbf{\small Keywords: Knowledge-Defined Networking, Machine Learning, Probing Interval, Software-Defined Networking, Traffic Monitoring}\\