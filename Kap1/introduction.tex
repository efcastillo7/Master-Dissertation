\chapter{Introduction}
\label{sec:introduction}

Software-Defined Networking (SDN) is a paradigm that promotes a flexible architecture for fast and easy configuration of network devices \cite{machado_2014:towards_SLA_Policy}. SDN is characterized by the network programmability, and the centralization of the control functions in the controller. Thanks to these features, the controller can perform fine-grained Network Management (NM) \cite{lopez_201:sdn-ml-wlans}. Nonetheless, such features are not enough to guarantee an appropriate network behavior when traffic reach unexpected levels \cite{herrera_2016:nfv_survey,ESTRADASOLANO2017150}. In this sense, Traffic Engineering (TE) is an important tool to assist the SDN operation \cite{ian_2014:a_road_map_sdn}. TE encompasses measuring and managing the network traffic, aiming at improving the utilization of network resources and enhancing the Quality of Service (QoS). TE requires an efficient Traffic Monitoring that allows the accurate and timely collection of flow statistics \cite{rendon2014:monitoring}.\\

SDN-enabled switches can measure different per-flow traffic statistics, such as byte and packet counters, duration per second, hard timeout, and lifetime. There are two ways in which the SDN controller can retrieve traffic statistics from the underlying switches: push-based and pull-based. In the push-based approach \cite{Yu_2013:flow_sense, suh_2014:OpenSample}, the controller passively receives reports from switches. This approach has some drawbacks. First, this method requires additional hardware and software components in the switches. Second, when the traffic varies dynamically, the switches frequently detect no-matching packets in the flow table and, as a result, massive statistical reports are sent to the controller. These massive reports can cause significant Control Channel Overhead (CCO) \cite{aslan_2016:impact} and an Extra CPU Usage of the Controller (CUC) \cite{su_2014:flowcover}. Third, the additional hardware and software elements can raise security issues \cite{dharsee_2017:software}. In the pull-based approach, the controller retrieves flow statistics from the switches using Read-State messages. This approach provides more flexibility that the push-based approach, since it can asynchronously communicate with the switches and request specific information, thus controlling the size of the statistical reports. Besides, this approach does not require changes in the software and hardware of switches. For these reasons, this master dissertation focus on the pull-based approach. \\

CCO can also appear in the  pull-based approach due to the probing interval. This overhead can lead to overload the controller (\textit{i.e.}, CUC) and significantly interfere with essential SDN functions, such as packet forwarding and route updating. Several research approaches have been conducted to deal with CCO and CUC \cite{chowdhury_2014:payless, raumer_2014:monsamp, van_2014:OpenNetMon, tahaei_2017:multi-objective, Tootoonchian_2010:opentm, Sun_2015:HONE, phan2017:sdn_mon, liao_2018:LLDP-looping,su_2014:flowcover,jose_2011:online_measurement, tangari_2017:decentralized_monitoring, Tangari_2018:adaptive_decentralized_monitoring, phan2017:adaptive_sdn_mon, tahaei_2018:cost_effective}. In particular, \cite{chowdhury_2014:payless, raumer_2014:monsamp, van_2014:OpenNetMon, tahaei_2017:multi-objective, Tootoonchian_2010:opentm} reduce CCO by using adaptive techniques, wildcards, threshold-based methods, and routing information at expenses of decreasing the Monitoring Accuracy (MA). Other approaches  diminish CCO by adding modules or modifying flow tables in the switches \cite{Sun_2015:HONE, su_2014:flowcover, phan2017:sdn_mon, liao_2018:LLDP-looping} and by adding distributed controllers \cite{jose_2011:online_measurement, tangari_2017:decentralized_monitoring, Tangari_2018:adaptive_decentralized_monitoring, phan2017:adaptive_sdn_mon, tahaei_2018:cost_effective}. Thus, these works reduce CCO but they increase the operational costs. Furthermore, it is noteworthy that, in pull-based solutions the trade-off between probing interval and MA has not been studied enough. Besides, few intelligent mechanisms have been proposed for optimizing such a trade-off by learning from network behavior. Therefore, the goal of this master dissertation is to investigate a practical approach (\textit{i.e.}, in terms of CCO, CUC, and MA) for intelligent probing in SDN. To achieve this goal, this dissertation raises the following research question.

\begin{center}
\textbf{How to intelligently probing SDN with a high accuracy and with a negligible network overhead?}\\
\end{center}

\textbf{Hypothesis:} The Machine Learning allows intelligent probing on SDN, improving the accuracy traffic monitoring and reducing the corresponding overhead.

%The below fundamental questions, associated with the afore raised hypothesis, guide the investigation conducted in this master dissertation. 
%\begin{itemize}
%    \item What is the performance, in terms of the overhead and monitoring accuracy, of solutions that use ML for network monitoring?
%    \item Which mechanisms could be employed to improve the performance of solutions that use ML for network monitoring? 
%\end{itemize}

%In order to overcome the shortcomings mentioned above, in this paper, we propose an approach called IPro. Our approach includes an architecture that follows the Knowledge-Defined Networking (KDN) \cite{mestres_2017:KDN} paradigm, an algorithm based on Reinforcement Learning (RL), and an IPro prototype. We use KDN to apply Machine Learning (ML) in SDN. In particular, RL allows optimizing the probing interval of IPro aiming at keeping CCO and CUC within target values. We evaluate the IPro prototype in an emulated environment by using a campus network topology. The evaluation results reveal that IPro keeps CCO less than 1.23\% and CUC less than 8\%. Furthermore, IPro has better MA ($\geq$ 90\%) than the the Periodic Probing Approach (PPA) \cite{tsai_2018:network_monitoring_review}. PPA is a pull-based approach that monitors the switches with a pre-defined probing interval.